\documentclass[12pt]{article}
\usepackage{amsmath}
\usepackage{graphicx}

\marginparwidth = 15pt
\oddsidemargin = 15pt
\title{Architectural Specifications}
\date{27/07/2013}

\begin{document}
  \maketitle
\begin{tabbing}
Group Name: \= HackJob Programmers
\\~
\\Project: Samsung DLNA Enabled Application
\\~
  \\* Group Members:
\end{tabbing}
	
	\begin{itemize}
	\item PA Janse van Rensburg - 29034354
	\item NJ Taljaard - 10153285
	\item B Kleinhans - 11080851
	\end{itemize}

Document Version 1.5

\newpage
\section{Architecture Requirements:}
\subsection{Architectural Scope:}

Responsibilities that need to be addressed by the software architecture:
\begin{itemize}
\item Data Persistence: Since session information and the data generated will be stored for longer than just the session, the storage of this data needs to be addressed by the software architecture. This need not be a database specifically, a secured directory per session to keep all the content generated should be sufficient.
\item Connectivity: An infrastructure needs to be in place for multiple users to communicate with a host, and a host to broadcast to multiple users, this could be achieved with the Samsung Allshare API.
\item Reporting: No reporting facility is required, though there should be an appropriate infrastructure for logging of the data that was sent and which devices sent it. This audit functionality could be achieved by cross-cutting the communication between the layers of the architecture, duplicating the messages sent as well as source and destination and logging in an appropriate audit table. 
\item Error Recovery: The architecture should allow for a certain degree of error recovery to be done. For example, when a user disconnects (due to signal loss or low battery), it should allow them to reconnect once their issue has been resolved. This could be achieved by setting an Active flag on each session to indicate if users should be prompted to reconnect if they got disconnected.
\item Security: The architecture should allow for secure communications/storage of session data, by encrypting session data and using some form of wireless encryption for the transmission of data (WEP/WPA2).



\end{itemize}
\newpage
\subsection{Quality Requirements:}
Quality Requirements for this project include the following:
\begin{center} \begin{tabular}[h]{|l|l|l|l|}
\hline
Req. Number & Source & Priority & Description \\
\hline
NFRQ-1 & Group & 1 & Accessibility: Each DLNA enabled device will be able to\\ & & & 
									 install and use the application, with \\ & & & 
									 the exception of devices used for media rendering \\ & & & 
									 purposes e.g. TV's, Audio Systems etc. Configuration\\ & & &  options with regards to screen contrast, font-size \\ & & & 
									 and other visibility options will be included to \\ & & & 
									 cater for handicapped users.\\
\hline
NFRQ-2 & Group & 1 & Availability: Each independent session (i.e. A session\\ & & & 
					 running on it's own hardware) is expected to have \\ & & & 
					 full uptime while in use. Services/sessions could be\\ & & & 
					  restarted in between meetings/events,but during a\\ & & & 
					   session it needs to be available nearly 100\% of the time. \\
\hline
NFRQ-3 & Group & 2 & Efficiency: Application will not hinder device functionality \\ & & & 	
					in the sense of memory usage being kept to a minimum, \\ & & & 
					as well as space and CPU utilization. Application will \\ & & & 
					 also need to be light with regards to power consumption\\ & & & 
					  while in use. *Testing will need to be done to determine\\ & & & 
					   the exact figures. \\
\hline
NFRQ-4 & Group & 2 & Effectiveness: With regards to interface interaction, \\ & & & 
					sub-second response times are expected for all functionality.\\ & & & 
					 This excludes file transfers within a session. \\
\hline
NFRQ-5 & Group & 1 & Extensibility: A modular structure will be utilized to\\ & & & 
					 ensure maximum plug-ability and re-use / addition \\ & & & 
					 of system components. \\
\hline
NFRQ-6 & Group & 1 & Platform Compatibility: Application will only be\\ & & & 
					 compatible with Android version “Ice-cream Sandwich”\\ & & & 
					  and later. \\
\hline
NFRQ-7 & Group & 3 & Scalability: A session needs to be scalable up to at \\ & & & 
					most 50 concurrent users, with 4 hosts and handle the \\ & & & 
					peak transfers of these users. That is, with a max \\ & & & 
					file size of 5MB, 20MB needs to be transmitted to \\ & & & 
					 the 50 users, as well as the server in a reasonable\\ & & & 
					  time frame (20 to 30 seconds). \\
\hline
\end{tabular} \end{center}
\begin{center} \begin{tabular}[h]{|l|l|l|l|}
\hline
Req. Number & Source & Priority & Description \\
\hline
NFRQ-8 & Group & 3 & Security: Users will be required to authenticate themselves \\ & & & 
					to a session by means of a login for the application, and \\ & & & 
					additionally a password will be setup by the host to\\ & & & 
					 prevent unauthorized access to private sessions. \\ & & & 
					  Encryption will be utilized for the wireless access\\ & & & 
					 as well as user login details and files being transmitted.\\
\hline
NFRQ-9 & Group & 1 & Usability: Final usability will be judged by some form of\\ & & & 
					 User Experience Tests, this application needs a high degree\\ & & & 
					  of usability and should be easily learn-able with an \\ & & & 
					  intuitive interface. \\
\hline
NFRQ-10 & Group & 1 & Reliability: Max users per session will be implemented\\ & & & 
					 to maintain session reliability. In addition, a file size \\ & & & 
					 cap of approximately 5 – 10mb will be placed on session\\ & & & 
					  sharing to avoid slowdowns in efficiency. Only 4 \\ & & & 
					  concurrent hosts will be allowed. \\
\hline
\end{tabular} \end{center}
\newpage
\subsection{Integration and access channel requirements:}
This software product will, for the time being, be a stand-alone product, that will run on a server with an access-point, thus there will be no integration into any software systems other than using the Samsung Allshare API.
Possible integration into social network software (i.e. Facebook, Twitter etc..) may be looked into as a "nice-to-have" feature, but is currently not part of the scope of our project.

\subsubsection{Channel requirements: } 
There will be 2 major access channels for this project:
\begin{itemize}
\item Guest/Host Access via a mobile DLNA enabled device. (Samsung Tablets, Smartphones, PC's).
\item A host only interface to set up the server-side preferences of the\\ program.
\end{itemize}

\subsubsection{Access Channels:}
An additional access channel that is required is the availability through the Samsung Apps store for users to get access to the application.

\newpage
\subsection{Architectural Constraints:}
Since this is an open-scoped project, very few constraints have been placed on this project,
constraints that have been placed on this project include the following:
\begin{itemize}
\item Programming Language: Android/Java with support from Android version 4 up to and including 4.3.2.
\item Device support: All Samsung DLNA devices using Android version 4.0.3 and above.
\item API's: Use of Samsung Allshare API for integration with DLNA enabled devices.
\item Framework: No development framework has been enforced by the client.
\item Reference Architectures: No reference architecture has been specified by the client.
\item Miscellaneous: Client has specified that this project should serve as a development basis for further extensions by Samsung.
\end{itemize}

\newpage
\section{Architectural Pattern/s:}
A possible architectural pattern to be used for this project:
\\ \textbf{Event Driven Architecture:}
The Event-Driven Architecture has 4 logical layers which handle the sensing of an event, then the reactions that follow that event. 
A simple events driven architecture could be used for this project since:
\begin{itemize}
\item Coincides with the general structure of an Android program, and thus is easier to draw parallels between the functionality and the design.
\item Our hand-held host and guest devices will generate events/activities upon which is reacted by the server side software.
\item The server will be the event processing engine that handles the storage of the files that were transmitted and logs the required user information.
\item The high degree of heterogeneity of the events in this system could allow for the implementation of this pattern. Nearly every event in this system will involve a file transfer or a variation thereof. 
\end{itemize}
The different elements in this Architectural Pattern could be realised as follows:
\begin{itemize}
\item \textbf{Event Generator}: Hand-held host and guest devices that share information.
\item \textbf{Event Channel}: The wireless network (TCP/IP) that connects these devices through the Samsung Allshare API.
\item \textbf{Event Processing Engine}: The server that responds to the events generated by the hand-held devices.
\item \textbf{Downstream Event-Driven Activity}: Will be updated to the hand-held devices.
\end{itemize}


\newpage
\section{Architectural Strategies:}
Architectural strategies that could be followed to help with scalability as well as performance:
\begin{itemize}
\item Database access should be monitored so that no locking occurs when only read actions are done (This will be handled by the RDBMS).
\item Device interaction should be asynchronous so that other usage of the application can continue (example, when a file is being requested, the guest should still be able to view what the host is showing.)
\item A thread pool should be set up for use when multiple guests/hosts connect to a server, to allow for better resource management.
\item The design should aim to keep a stateless as possible approach, i.e. keeping most of the state stored on the guest devices, so as to avoid having to keep track of all the guest information on the server.

\end{itemize}


\section{Use of reference architectures and\\ frameworks:}
It does appear that Android itself is seen in some circles as a development framework since it sets forth a rather strict way in which programs are to be structured (namely, activity based, event driven type structure, with specific use of XML to define layouts and user interfaces.).
That is currently the only framework we plan on using in this project, as well as making use of the Samsung AllShare API.
A possible reference architecture we may use is the \textbf{Event-Driven architecture} for the reasons stated earlier.

\newpage
\section{Integration and access channels:}
As previously stated no integration into existing systems is to be done.
Access channels will include the 2 previously stated main access channels.

\section{Technologies:}
Possible technologies to be used:
\begin{itemize}
\item Android/Java for programming the application, on Windows based platforms.
\item Samsung AllShare API for interaction with other DLNA enabled\\ devices.
\item MySql for possible database needs, used with JDBC for integration with Android platform.
\item Some form of wireless encryption. (possibly WEP)
\item Github for distributed version control.
\end{itemize}



\end{document}