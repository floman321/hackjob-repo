\documentclass[12pt]{article}
\usepackage{amsmath}
\usepackage{graphicx}
\title{Vision and Scope}
\date{08/05/2013}

\begin{document}
  \maketitle
\begin{tabbing}
Group Name: \= HackJob Programmers
\\~
\\Project: Samsung DLNA Enabled Application
\\~
  \\* Group Members:
\end{tabbing}
	
	\begin{itemize}
	\item PA Janse van Rensburg - 29034354
	\item NJ Taljaard - 10153285
	\item B Kleinhans - 11080851
	\end{itemize}

Document Version 1.0

\newpage
\section{Business/Client Perspective :}
\subsection{Background}
There has been a need for some software that allows people to easily manage content that was exchanged/generated during social/business events. These business events usually make use of paper copies of the work that is to be discussed and/or electronic copies that need to be distributed to the attending personnel, as well as possible presentations or media that needs to be displayed during a meeting session.
\\During a social event, people usually struggle to consolidate all the media generated during the event on various of the attendees' social network pages and/or emails with the media attached.
\\ Our project will aim to satisfy these needs as well as simplify the usage of the AllShare API, by creating a simple to understand, easy to manage and fun to use application to allow specific people to access a user group, contribute media to that user group and possibly later synchronize the content with their devices after the event.

\subsection{Business Opportunity}
If this product were commercialized, it would serve as a motivation for people to move to DLNA enabled devices to use these products. This product can specifically be used in for example, a boardroom situation where specific people are attending and require documentation, and where, for example a presentation or other media needs to be displayed.

\subsection{Customer and Market requirement}
It's not clear whether this product will be marketed as of yet.
There does exist software that does allow people to share documentation etc.. But none (that we could find within and hour on Google) make use of the DNLA technology or attempt to incorporate the entertainment aspect that features in our product.

\newpage
\section{Vision:}
The vision that we have for this project is that at the end there will be a useful, fully functional product that is up to an acceptable and usable standard. This will be our guideline when making decisions w.r.t. design and any implementation details.

\section{Scope and Limitations:}
The client has accepted our idea but wants to change the use cases for the product, to make it more usable for their market. Our initial proposal's ideas will still be used, just the way they are used will be changed once we get into contact with our client again. Our product will be aimed towards Android based devices that support DLNA.

\section{High-level non-functional requirements:}
Our aim will be to make this product compatible with multiple devices and different platforms

\subsection{Authentication:}
	\begin{itemize}
	\item Access control will be implemented in the form of WEP Authentication to the application DLNA enabled network.
	\item Simple login, logout access control to the application to protect local user app data.
	\item Therefore only users with valid authentication will be able to access the application and the network.
	\end{itemize}

\subsection{Authorization:}
	\begin{itemize}
	\item A model of privileges will be compiled according to specific user roles
	\item These privileges will be assigned at logon / logoff.
	\end{itemize}

\subsection{Scalability:}
	\begin{itemize}
	\item Application should cater for up to 50 connected users to any specific server network.
	\item In time application could be extended to be platform independent across a range of devices
	\end{itemize}

\subsection{Auditability:}
	\begin{itemize}
	\item Actions on the system should be logged.
	\item Will be further discussed with client
	\end{itemize}

\section{Project Success Factors}
\subsection{Driver}
	The client would prefer a product where all the basic features are implemented and working 100 percent instead of having a wide variety of features not working successfully. The implementation of the project is of greater importance to the client than creating a project for mere marks. They encourage a usable product at the end of this endeavour. A connection between multiple 		devices needs to be possible. The need for drag drop gestures as well as multi-gestures for sharing between devices.
\subsection{Constraint}
	Share of visual and audio files. Usability in the working and social environment. To be able to access the shared data of the connected devices.
\subsection{Important}
	Increase in sales of Samsung devices. Companies using Samsung/Android devices in working environment. Making Samsung devices more popular among the youth of society for the social connectivity. Using our product as a baseline for further improving technology.
\subsection{Nice-to-have's}
	A non-configuration connection. Support a wide variety of platforms.

\end{document}