\documentclass[12pt]{article}
\usepackage{amsmath}
\usepackage{graphicx}

\marginparwidth = 15pt
\oddsidemargin = 15pt
\title{Requirements Specifications}
\date{27/07/2013}

\begin{document}
  \maketitle
\begin{tabbing}
Group Name: \= HackJob Programmers
\\~
\\Project: Samsung DLNA Enabled Application
\\~
  \\* Group Members:
\end{tabbing}
	
	\begin{itemize}
	\item PA Janse van Rensburg - 29034354
	\item NJ Taljaard - 10153285
	\item B Kleinhans - 11080851
	\end{itemize}

Document Version 1.2

\newpage
\section{Introduction:}
\subsection{Purpose:}

The purpose of this document is to formulate an agreement between the developers of Team HackJob from the University of Pretoria and Samsung about the requirements of the DLNA enabled application, provisionally named “OmniShare”.  This document ensures that the scope of the project is clear and understood by all parties so that the requirements set up in this document form the basis of further phases. Due to the nature of the software
\\ development process i.t.o. requirement finalization, this document acts as an official contract between both parties.

\subsection{Document Conventions:}
\begin{itemize}
\item LaTeX.
\item Crow's foot notation for ERD's.
\item UML 2.0 for UML Diagrams.
\end{itemize}

\subsection{Project Scope:}
The scope of the Omnishare project can be summarized as a software solution that makes use of the Samsung AllShare API and DLNA enabled technologies to:
\begin{itemize}
\item Allow for the creation and maintenance of a network of AllShare enabled, SAMSUNG authenticated devices  that will allow for simplified, mass sharing of data and utilization of digital mediums (such as
\\ televisions for display purposes) in an efficient, simplified manner.
\item  Create a home entertainment network to allow for centralization of data and control of DLNA enabled devices.
By formulating this foundation, the application can later be expanded to be cross platform and 
\\device compatible.  Later, multiple sessions (access points) can be made available and managed per server, so server can host multiple parallel sessions.
\end{itemize}

\subsection{References:}
Andrew Syrett, by means of requirements elicitation and formal meetings.

\subsection{Related Documents:}
Related documents to the current one, are the following :
\begin{itemize}
\item Design Document
\item Architectural Specification Document
\end{itemize}

\newpage
\section{System Description:}

System will consist of 3 main parts, namely Sessions, Device Control and Media Handler.
\subsection{Sessions:}
Server PC will broadcast a wireless access point in the form of an ID to which devices can connect, which is found by a device by utilizing the currently implemented device scanning technology.  A session will be created, as well as a directory within a storage directory for the session’s media when a device successfully connects to the access point and creates a “New Meeting” via the application interface.  This device will be the host for the session.  Users can then join this session by scanning and connecting to the created session.  Users will be referred to as “Guests” by the application and connecting to the session will require authentication.  Relevant details to the authentication process will be encrypted.
Data persisted to the directory on the server will take place by cross cutting communication and storing a copy in the session’s created media directory.  This data will only be kept for a decided upon time interval.  To accomplish the above session functionality the server in this case will be a dedicated machine.  Sessions will provide sub-second response times.  A session is only deemed complete once a host sets a completed flag to the meeting, otherwise the session is kept and notification to reconnect to the session will be offered to the host device.  The server will periodically check that connected devices are active, and active devices will be removed from the list of connected devices to the session.  A device is declared inactive if server receives no response for a test packet for example if signal is lost.

\subsection{Device Control:}
This part will handle all issues regarding the actual devices.  
With regards to visibility, Guests will be blind to other devices on the network and available devices, whereas the Host will be fully aware of the connected devices as well as available devices.  Media devices will need to provide mutually exclusive access.  Once such a device is claimed by the host device it should not appear in a device scan.  On the host device, other devices will be grouped according to type and functionality.  The application will need to provide independent media device control to the host.  This control will need to be able to be reassigned to another device provisionally.  The Host will also need to be able to add additional hosts and grant the relative rights to these devices.  The Host device can broadcast media to all other devices connected to the session.
Guest devices will have a SUGGEST function, which will allow them to send documents to the host for redistribution.

\subsection{Media Handler:}
This part will handle the persistence and transmission of files from the devices to the server and vice verse.
It will be used internally by the application. It will set up a connection to the server from the device then transmit the files required and store them in the appropriate place.

\newpage

\section{Quality Requirements:}
For quality / non-functional requirements, please view our architectural
\\ specification document.

\newpage
\section{Functional Requirements:}
\begin{center} \begin{tabular}[h]{|l|l|l|l|}
\hline
Req. Number & Source & Priority & Description \\
\hline
FRQ-1 & Group & 1 & DLNA Enabled Android devices need to connect to a Wi-Fi\\ & & &
					based server.  This server will be a machine on the premises\\ & & &
					broadcasting a regular WPA2 encrypted SSID for\\ & & & 
					 connection purposes.\\
\hline
FRQ-2 & Group & 1 & Session must be running on a server, ready for use by a host\\ & & & 
					device.\\
\hline
FRQ-3 & Group & 1 & Host Device needs to be assigned to session. Later, \\ & & & 
					additional hosts must be added to a session.\\
\hline
FRQ-4 & Group & 1 & Multiple Guest devices need to be able to connect to\\ & & & 
					a started session.\\
\hline
FRQ-5 & Group & 1 & Copies of documents exchanged / generated during \\ & & & 
					session should be persisted to the server in the \\ & & & 
					session's directory. (Party Mode and Regular Mode)\\
\hline
FRQ-6 & Group & 1 & Guests should be able to “Suggest” a file to the host to be\\ & & & 
					 redistributed by means of a flag and notification.\\
\hline
FRQ-7 & Group & 1 & Multiple Multimedia Rendering devices (TV's etc..) \\ & & & 
					should be visible to the host device via the network.\\
\hline
FRQ-8 & Group & 1 & Local storage of meeting details and references\\ & & & 
					 of files for meetings.\\
\hline
FRQ-9 & Group & 2 & Multimedia Rendering devices should be able to display \\ & & & 
					host's selected content.\\
\hline
FRQ-10 & Group & 5 & PartyMode: Guests should be able to send files for \\ & & & 
					display to a Multimedia Rendering device (Send / Stream)\\
\hline
FRQ-11 & Group & 5 & PartyMode: Guests should be able to select files\\ & & & 
					 to be persisted during a session.\\
\hline
FRQ-12 & Group & 3 & Transmission of data needs to be encrypted.\\
\hline
FRQ-13 & Group & 3 & Storage of data needs to be secure.\\
\hline
FRQ-14 & Group & 3 & Guests need to be authenticated to join a session.\\
\hline
FRQ-15 & Group & 4 & If a new guest arrives late / got disconnected,\\ & & & 
					 current session data should be pushed to the \\ & & & 
					 new device.\\
\hline
FRQ-16 & Group & 5 & Logging of transfers (I.e. What file, who sent it,\\ & & & 
					 when it was sent.)\\
\hline



\end{tabular} \end{center}

\section{External Interface Requirements:}
Currently the only interface's our application have to deal with are the
\\following:
\begin{itemize}
\item Mobile Based Interface: For devices to use the facilities of our application.
\item Server Based Interface: To configure settings related to the server we may be using.
\end{itemize}

\end{document}